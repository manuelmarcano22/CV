% LaTeX Curriculum Vitae Template
%
% Copyright (C) 2004-2009 Jason Blevins <jrblevin@sdf.lonestar.org>
% http://jblevins.org/projects/cv-template/
%
% You may use use this document as a template to create your own CV
% and you may redistribute the source code freely. No attribution is
% required in any resulting documents. I do ask that you please leave
% this notice and the above URL in the source code if you choose to
% redistribute this file.
%
% If you have never used LaTeX before, talk you your advisor about
% how to compile this into a PDF.

\documentclass[letterpaper,10pt]{article}

\usepackage{hyperref}
		\usepackage{amssymb}
\usepackage[hmargin=1.5cm,vmargin=1.5cm]{geometry}
\usepackage{geometry}
% Comment the following lines to use the default Computer Modern font
% instead of the Palatino font provided by the mathpazo package.
% Remove the 'osf' bit if you don't like the old style figures.
\usepackage[T1]{fontenc}
\usepackage[sc,osf]{mathpazo}
\usepackage[utf8]{inputenc}
\inputencoding{latin1}
\usepackage{comment}
\usepackage{multicol}
% Set your name here
\def\name{Manuel Pichardo Marcano}

% Replace this with a link to your CV if you like, or set it empty
% (as in \def\footerlink{}) to remove the link in the footer:
\def\footerlink{}

% The following metadata will show up in the PDF properties
\hypersetup{
  colorlinks = true,
  urlcolor = black,
  pdfauthor = {\name},
  pdfkeywords = {CV, physics},
  pdftitle = {\name: Curriculum Vitae},
  pdfsubject = {Curriculum Vitae},
  pdfpagemode = UseNone
}

\geometry{
  body={6.5in, 8.5in},
  left=.5in,
  top=.6in
}

% Customize page headers
\pagestyle{myheadings}
\markright{\name}
\thispagestyle{empty}

% Custom section fonts
\usepackage{sectsty}
\sectionfont{\rmfamily\mdseries\Large}
\subsectionfont{\rmfamily\mdseries\itshape\large}

% Other possible font commands include:
% \ttfamily for teletype,
% \sffamily for sans serif,
% \bfseries for bold,
% \scshape for small caps,
% \normalsize, \large, \Large, \LARGE sizes.

% Don't indent paragraphs.
\setlength\parindent{0em}



% Make lists without bullets
 \usepackage{enumitem}
\usepackage{bbding}
\usepackage{tikz}
\usepackage{pifont} 
%\renewenvironment{itemize}{
  %\begin{list}{}{
    %\setlength{\leftmargin}{1.5em}
  %}
%}{
  %\end{list}
%}

\begin{document}

% Place name at left
\begin{huge}
\name
\end{huge}

% Alternatively, print name centered and bold:
%\centerline{\huge \bf \name}

%\vspace{0.25in}

%\section*{Contact Information} \hrule


\vspace{.3 cm}



\begin{minipage}{0.6\linewidth}
  \href{www.manuelpm.me}{manuelpm.me} \\
Fisk University \\
  Departamento de Ciencias F\'isicas y de la Vida \\
  Nashville, TN 37208\\
%Phone: +1 809 757 7796 \\
correo electr\'onico:  \href{mailto:mmarcano@fisk.edu}{mmarcano@fisk.edu}\\
 %Email: \href{mailto:manuelpichardom@gmail.com}{ manuelpichardom@gmail.com}\\
 %Pronouns: he/him/his 
\end{minipage}
%\begin{minipage}{0.45\linewidth}
  %\begin{tabular}{ll}
    %Phone: & (757) 555-5555 \\
    %Fax: &  (757) 221-3540 \\
    %Email: & \href{mailto:me@email.wm.edu}{\tt me@email.wm.edu} \\
  %\end{tabular}
%\end{minipage}



\section*{Intereses de Investigaci\'on}

\hrule
\vspace{.3 cm}


Objetos Compactos; Objetos Compactos en C\'umulos Globulares; Binarias Enanas Blancas; Variables Catacl\'ismicas; AM~CVns; Estrellas Variables, Sistemas Binarios, An\'slisis de Datos.

\section*{Experiencia de Investigaci\'on Posdoctoral}

\hrule
\vspace{.3 cm}
\begin{itemize}[label=$\blacktriangleright$]


 \item \emph{Univesidad Fisk/Vanderbilt, Nashville, TN}\\
 Becario Postdoctoral  \\
 September 2023 - Presente



 \item \emph{American Museum of Natural History, New York, NY}\\
 Beca Postdoctoral Kalbfleisch   / Investigador Asociado \\
  September 2022 - Presente\\
 

 

\end{itemize}



\section*{Educaci\'on}

\hrule
\vspace{.3 cm}


\begin{itemize}[label=$\blacktriangleright$]



  \item \emph{PhD., F\'isica}  \\
   Universidad Tecnol\'ogica de Texas (Texas Tech University) , 2018-2022
      \begin{itemize}[label=\ding{27}]
     Tesis: Variabilidad de la Acreci\'on de binarias de enanas blancas y otros objetos compactos  \\
     Asesores: Profesor Thomas J. Maccarone y Professora Liliana E. Rivera Sandoval 
      \end{itemize}





  \item \emph{Maestr\'ia en F\'isica}  \\
   Universidad Tecnol\'ogica de Texas (Texas Tech University), 2016-2018



  \item \emph{M\'aster Europeo en Ciencias y Tecnolog\'ias Espaciales} 
  \begin{itemize}[label=\ding{27}]
      \vspace{-.05cm}
      \item Tesis: Observaciones con MUSE de los objetos compactos en NGC 6397 \\
      M.S., Ciencias y Tecnolog\'ias Espaciales, Lule\r a tekniska universitet  (2014 - 2016) \\
      M.S., Astrof\'isica, Universit\'e Paul Sabatier Toulouse III, Mention Assez Bien (2014 - 2016) \\
      Asesores: Dr. Natalie Webb and Dr. Sebastien Guillot
  \end{itemize}




  \item \emph{Licenciatura en F\'isica}  \\
   Universidad Estatal de Utah (Utah State University), Cum Laude (2014)

 % \item Ph.D. Physics, College of William and Mary, expected 2010.
\end{itemize}





%\clearpage


\section*{Publicaciones}%\& Preprints}
\hrule
\vspace{.3 cm}
\subsection*{Art\'iculos Cient\'ificos en Revistas indexadas }
\begin{itemize}[label=$\blacktriangleright$]

%\subsubsection*{Refereed}

                \item \textbf{Pichardo Marcano, M.}, Rivera Sandoval, L. E., Maccarone, T. J., Rohrmann, R. D., Heinke, C. O., Belloni, D., Althaus, L. G., \& Bahramian, A. (2023). Una candidata enana blanca magn\'etica con n\'ucleo de helio en el c\'umulo globular NGC 6397 (A candidate magnetic helium-core white dwarf in the globular cluster NGC 6397), \href{https://ui.adsabs.harvard.edu/abs/2023MNRAS.521.5026P/abstract}{Monthly Notices of the Royal Astronomical Society, 521, 5026.}


        %%l (%Y). %T,%J, %V, %p.\n



        
        \item \textbf{Pichardo Marcano, M.}, Rivera Sandoval, L. E., Maccarone, T. J., \& Scaringi, S. (2021). TACOS: Sondeo de Estallidos en las AM~CVn(TACOS: TESS AM CVn Outbursts Survey),Monthly Notices of the Royal Astronomical Society, 508, 3275.


        
                
            \item \textbf{Pichardo Marcano, M.}, Rivera Sandoval, L. E., Maccarone, T. J., Zhao, Y., \& Heinke, C. O. (2021). \href{https://ui.adsabs.harvard.edu/abs/2021MNRAS.503L..51P/abstract}{Per\'iodo orbital de 2 d\'ias para un candidato a p\'ulsar de milisegundo de espalda roja en el c\'umulo globular NGC 6397(A 2-d orbital period for a redback millisecond pulsar candidate in the globular cluster NGC 6397), Monthly Notices of the Royal Astronomical Society, 503, L51.}
        
        
        \item Schwope, A., Buckley, D. A. H., Kawka, A., König, O., Lutovinov, A., Maitra, C., Mereminskiy, I., Miller-Jones, J., \textbf{Pichardo Marcano, M.}, Rau, A., Semena, A., Townsend, L. J., \& Wilms, J. (2022). Identificaci\'on de SRGt 062340.2-265751 como una variable catacl\'ismica similar a una nova, brillante y fuertemente variable (Identification of SRGt 062340.2-265751 as a bright, strongly variable, novalike cataclysmic variable),Astronomy and Astrophysics, 661, A42
        
        

        
    \item Rivera Sandoval, L. E., Maccarone, T. J., \& \textbf{Pichardo Marcano, M.} (2020). Una superexplosi\'on de un a\~{n}o de duraci\'on de una binaria enana blanca ultracompacta revela la importancia de la irradiaci\'on de estrellas donantes (A Year-long Superoutburst from an Ultracompact White Dwarf Binary Reveals the Importance of Donor Star Irradiation), The Astrophysical Journal, 900, L37. \\

    \item Zhao, Y., Heinke, C. O., Tudor, V., Bahramian, A., Miller-Jones, J. C. A., Sivakoff, G. R., Strader, J., Chomiuk, L., Shishkovsky, L., Maccarone, T. J., \textbf{Pichardo Marcano, M.}, \& Gelfand, J. D. (2020). El sondeo MAVERIC: un p\'ulsar oculto y un candidato a agujero negro en las im\'agenes de radio ATCA del c\'umulo globular NGC 6397,(The MAVERIC survey: a hidden pulsar and a black hole candidate in ATCA radio imaging of the globular cluster NGC 6397), Monthly Notices of the Royal Astronomical Society, 493, 6033. \href{10.1093/mnras/staa631} \\


\item  S.W. McIntosh, W.J. Cramer, \textbf{M. Pichardo Marcano}, R.J. Leamon. \emph{La detecci\'on de ondas tipo Rossby en el Sol(The detection of Rossby-like waves on the Sun)}. Nature Astron. 1, 0086 (2017). \href{http://dx.doi.org/10.1038/s41550-017-0086}{\tt doi:10.1038/s41550-017-0086} \\

%\subsubsection*{Non-refereed}

%\item Maccarone, T. J., Beardmore, A., Mukai, K., Page, K., \textbf{Pichardo Marcano, M.}, \& Rivera Sandoval, L. (2021). X-ray Pulsations from Nova Her 2021,The Astronomer's Telegram, 14776, 1.



%\item \textbf{Pichardo Marcano, M.} (2020). A period of 3.9 hours in the TESS light curve of SRGt 062340.2-265715,The Astronomer's Telegram, 14222, 1.

%\item  Jeffrey S. Hazboun, \textbf{Manuel Pichardo Marcano} and Shane L. Larson. \emph{Limiting alternative theories of gravity using gravitational wave observations} \\
 preprint \url{http://arxiv.org/abs/1311.3153} \\


%submitted to Classical \& Quantum Gravity (2014); \href{http://arxiv.org/abs/1311.3153}{\tt arxiv/1311.3153}
\end{itemize}

\subsection*{Charlas y P\'osters Acad\'emicos}

\begin{itemize}[label=$\blacktriangleright$]

\item \textit{Variabilidad de las binarias de enanas blancas acretantes} \\ Charla. V Congreso Internacional de la Sociedad Dominicana de F\'isica. 2024.


\item \textit{Sondeo MARES: Remanentes magn\'eticos en simbi\'oticas} \\ P\'oster. XVII Reuni\'on Regional Latinoamericana de la UAI . 2023.

\item \textit{ Sondeo SCOVaS: Estudio de objetos compactos y estrellas variables}  \\ Charla. Two in a Million - The interplay of binaries and star clusters. 2023


\item \textit{ Sondeo SCOVaS: Estudio de objetos compactos y estrellas variables}  \\ Charla. The Transient and Variable Universe. 2023


\item \textit{ Variabilidad de binarias de enanas blancas y otros objetos compactos.} \\ Ponente invitado. Carnegie Observatories. 2023

\item \textit{Una candidate a enana blanca magn\'etica con n\'ucleo de helio en el c\'umulo globular NGC 6397} \\ Charla. 241 Reuni\'on de la Sociedad Astron\'omica Estadounidense. 2023.

\item \emph{Una b\'usqueda de estrellas variables y binarias compactas en c\'umulos globulares con el telescopio espacial Hubble.} \\ Ponente invitado. Seminario del Departamento de F\'isica de la Universidad Trinity. San Antonio, TX. 2022

\item \textit{TACOS: Sondeo de Estallidos en las AM~CVn} \\ Charla. AM~CVn 4.5 Meeting. September 2022.

\item \textit{Variabilidad de binarias de enanas blancas y otros objetos compactos.} \\ Charla de Tesis Doctoral. 240 Reuni\'on de la Sociedad Astron\'omica Estadounidense. 2022.


\item \emph{Una b\'usqueda de estrellas variables y binarias compactas en c\'umulos globulares con el telescopio espacial Hubble.} \\Ponente invitado. Seminario del Departamento de F\'isica de La Universidad de Texas Valle del R\'io Grande. 2022


\item \textit{Resultados del sondeo TACOS de Estallidos de las AM CVn} \\ Charla. TESS Science Team Meeting $\#26$. 2021.


\item \textit{Astronom\'ia en Ultravioleta.} \\
Ponente Invitado. Escuela de Verano de Alpha-Cen. 2021.

\item \textit{Una b\'usqueda de estrellas variables y binarias compactas en c\'umulos globulares con el telescopio espacial Hubble.}\\ Charla. Reuni\'on Anual y Asamblea General de Alpha-Cen. 2020


\item \textit{Sondeo SCOVaS: Estudio de objetos compactos y estrellas variables}\\ P\'oster. 36th Annual New Mexico Symposium. 2020

\item \textit{\href{https://manuelpm.me/TexasAPS2019/}{"Binarias compactas en el c\'umulo globular NGC~6397"}} \\Charla. Texas APS Section. Lubbock, TX. 2019



\item \textit{\href{http://manuelpm.me/papers/MUSEP\'oster.pdf}{"Observaciones de la unidad de campo integral MUSE de los objetos compactos en el c\'umulo globular NGC 6397"}} \\P\'oster. Stellar Remnants at the Junction. Junction, TX. 2016


%\item \textit{\href{http://manuelpm.me/papers/P\'osterhao.pdf}{"Big Flare Hunting"}} \\P\'oster. Summer Program in Solar and Space Physics. Boulder, CO. 2013


\item \textit{\href{http://manuelpm.me/papers/P\'osterinsar.pdf}{"Medici\'on de las fluctuaciones del nivel del agua de dos humedales conectados en la Rep\'ublica Dominicana utilizando InSAR"}} \\P\'oster. SHPE National Conference. Dallas, TX. 2012

\item \emph{"Medici\'on de las fluctuaciones del nivel del agua de dos humedales conectados en la Rep\'ublica Dominicana utilizando InSAR"}\\
Charla. Coloquio Departamento de Física de la Universidad Estatal de Utah. 2012
\end{itemize}
\begin{comment}
\subsection*{Media Coverage}

\begin{itemize}[label=$\blacktriangleright$]

\item \hyperref[https://phys.org/news/2017-03-planetary-earth-sun.html]{Planetary waves, first found on Earth, are discovered on Sun (phys.org)}

\end{itemize}
\end{comment}
\section*{Propuestas de Investigaci\'on}       

\hrule
\vspace{.3 cm}

%\subsection*{CoI}

\begin{itemize}[label=$\blacktriangleright$]
      \item   Sat\'elite de estudio de exoplanetas en tr\'ansito (TESS)(Transiting Exoplanet Survey Satellite (TESS)): \emph{COPAS: Objetos compactos y estudio de la f\'isica de la acreci\'on ("COPAS: Compact Objects And The Physics Of Accretion Survey")} Programa Grande. (Co-Investigador)
  \end{itemize}

\begin{itemize}[label=$\blacktriangleright$]
      \item   El Observatorio G\'eminis:  Cycle 2023B (16 hrs). (Co-Investigador)
  \end{itemize}




  \begin{itemize}[label=$\blacktriangleright$]
      \item Telescopio espacial Hubble (HST):  Propuesta de para investigaci\'on de datos archivos. \emph{Una b\'usqueda de estrellas variables y binarias compactas en c\'umulos globulares con el telescopio espacial Hubble.} (Co-Investigador)
  \end{itemize}
  
  

\begin{itemize}[label=$\blacktriangleright$]
      \item   Sat\'elite de estudio de exoplanetas en tr\'ansito (TESS)(Transiting Exoplanet Survey Satellite (TESS)): \emph{"Estudiando los estallidos de AM~CVns con TESS (Studying the outbursts of AM~CVns with TESS)"} (Co-Investigador)
  \end{itemize}


  \begin{itemize}[label=$\blacktriangleright$]
        \item  Unidad de Campo Integral MUSE: \emph{La emisi\'on de hidr\'ogeno alfa de las binarias compactas del c\'umulo globular NGC 6397(The Hydrogen Alpha Disk Emission from the Compact Binaries of the Globular Cluster NGC 6397)} (Co-Investigador 1.5 h)
  \end{itemize}


\section*{Experiencia en Ense\~{n}anza y Mentor\'ia}

\hrule
\vspace{.3 cm}
\subsection*{Ense\~{n}anza}

%\hangindent=0.4cm  
\begin{itemize}[label=$\blacktriangleright$]
\item   Instructor graduado a tiempo parcial PHYS 1403/1404: Instructor de F\'isica introductoria basada en \'algebra. Universidad Tecnol\'ogica de Texas, 2017.
 
 \item   Ayudante de Docencia PHYS 1401/2401: Principios de F\'isica I/II. Monitor de los laboratorios  para la introducci\'on a la f\'isica basada en el c\'alculo. Universidad Tecnol\'ogica de Texas, 2016-2017.

 \item Becario docente de pregrado PHYS 2215: Laboratorio I de f\'isica para cient\'ificos e ingenieros - Universidad Estatal de Utah. 2013.
 
 
 
  \end{itemize}
  
  \subsection*{Mentor\'ia}

  
  \begin{itemize}[label=$\blacktriangleright$]
  
  \item Mentor del programa \emph{de tutor\'ia en investigaci\'on cient\'ifica} en el Museo Americano de Historia Natural. Mentor de 3 estudiantes de secundaria realizando investigaciones sobre aprendizaje autom\'atico en astronom\'ia. Septiembre 2022 - Junio 2023

          \item Mentor for \emph{Alpha-Cen: Asociaci\'on de Astrof\'isica en Centroam\'erica y el Caribe}. 2023.

   \item Mentor for \emph{El programa Centroam\'erica-Caribe en astrof\'isica (CENCA)}. CENCA es una experiencia de investigaci\'on remota para estudiantes universitarios. Primavera de 2021, oto\~{n}o de 2022/23
\end{itemize}
  
  \begin{comment}
  
\begin{itemize}[label=$\blacktriangleright$]

	\item \emph{Graduate Part-Time Instructor} PHYS 1403/1404: Algebra-based, Inquiry-based, Laboratory-based Introductory Physics Curriculum. TTU, 2017-Present.  
	
	

	
\item \emph{Teaching Assistant.}  PHYS 1401/2401: Principles of Physics I/II. Lead the laboratories and discussions for the calculus based introductory physics. TTU, 2016-2017.  
  	  

  \item \emph{Undergraduate Teaching Fellow.}  PHYS 2215: Physics for Scientists and Engineers Lab I -  Utah State, Fall 2013.
  
  
  
  
 
  
  
  \end{itemize}
  
  
  
  
  
  
  
  
  

% \section*{Computer \& Language Skills}      
% \emph{Languages}: Spanish (Native), English (Fluent), French (Intermediate)  \\
% \emph{Computer Languages}: Fortran, \LaTeX\\
% \emph{Software:}: Mathematica, MathCAD, Matlab



\end{comment}



\section*{Reconocimientos y Experiencias de Liderazgo}       

\hrule
\vspace{.3 cm}

\begin{itemize}[label=$\blacktriangleright$]


\item Estudiante de Doctorado Sobresaliente: \emph{Departamento de F\'isica y Astronom\'ia de la Universidad Tecnol\'ogica de Texas}. 2022


\item Beca de viaje FAMOUS: Beca para asistir a la 240 Reuni\'on de la Sociedad Astron\'omica Estadounidense. 

\item Co-organizador \href{https://comscicon.com/comscicon-en-espanol-2021}{Primera ComSciCon en Espa\~nol}. El taller de comunicaci\'on de la ciencia para estudiantes de posgrado.2021.




\item Beca Bucy : \emph{Departamento de F\'isica y Astronom\'ia de la Universidad Tecnol\'ogica de Texas. 2020-2021} %\\  % Need to check this     



%\end{itemize}

%\begin{itemize}[label=$\blacktriangleright$]

  \item Beca Erasmus Mundus: Beca de 2 a\~{n}os otorgada por la Comisi\'on Europea para realizar una maestr\'ia. % \\
  
\item Beca del Ministerio de Educaci\'on Superior, Ciencia y Tecnolog\'ia de la Rep\'ublica Dominicana para la licenciatura en la Universidad Del Estado de Utah.

  
%\end{itemize}

%\begin{itemize}[label=$\blacktriangleright$]

\item  Member: \emph{Sigma Pi Sigma}, La Sociedad de Honor de F\'isica % \\   
%\hspace{.2 cm}   $\blacktriangleright$  Vice-President: \emph{Society of Hispanic Professional Engineers.} 2011-2012\\
%\hspace{.2 cm}   $\blacktriangleright$  Outreach Coordinator: \emph{Society of Hispanic Professional Engineers.} 2012-2013 \\  
%\end{itemize}

%\begin{itemize}[label=$\blacktriangleright$]

\item Beca Lawrence  R. and Abeline Megill: \emph{Departamento de F\'isica de la Universidad del Estado de Utah.}   % Need to check this     
  \end{itemize}




\begin{comment}

\hangindent=0.4cm  

$\blacktriangleright$ Bucy Scholarship in Applied Physics: \emph{TTU Physics \& Astronomy Department. Fall 2020} \\  % Need to check this     
  $\blacktriangleright$  Erasmus Mundus Scholarship: 2-year scholarship awarded by the European Commission  \\    
 \hspace{.2 cm}    $\blacktriangleright$  Presidential Scholarship: Full 4-year scholarship awarded by the Dominican Government \\    
\hspace{.2 cm}   $\blacktriangleright$  Member: \emph{Sigma Pi Sigma}, The Physics Honor Society  \\   
%\hspace{.2 cm}   $\blacktriangleright$  Vice-President: \emph{Society of Hispanic Professional Engineers.} 2011-2012\\
%\hspace{.2 cm}   $\blacktriangleright$  Outreach Coordinator: \emph{Society of Hispanic Professional Engineers.} 2012-2013 \\  
\hspace{.2 cm}   $\blacktriangleright$   Lawrence  R. and Abeline Megill Scholarship: \emph{USU Physics Department}   % Need to check this     
\end{comment}



%\section*{Computer Programming}
\section*{Habilidades T\'ecnicas}
\hrule
\vspace{.3 cm}
    \begin{multicols}{2}


%\begin{itemize}[label=$\blacktriangleright$]
	%\item GitHub projects and contributions: %\href{https://github.com/manuelmarcano22}{https://github.com/manuelmarcano22}.
  %\end{itemize}


\begin{itemize}[label=$\blacktriangleright$]
	\item \emph{Lenguajes de programaci\'on}: Python, bash y R. 
\end{itemize}



  \begin{itemize}[label=$\blacktriangleright$]
  \item \emph{Sistemas operativos}: GNU/Linux, OS X, Windows
    \end{itemize}

  \begin{itemize}[label=$\blacktriangleright$]
          \item \emph{Programas Inform\'aticos}: IRAF/PyRAF, DOLPHOT, SAOImageDS9/JS9
    \end{itemize}

  \begin{itemize}[label=$\blacktriangleright$]
          \item \emph{Sistemas de control de versiones}: Git
    \end{itemize}
  

    
 %   \begin{itemize}[label=$\blacktriangleright$]
 %           \item \emph{Networking Tools/Services}: SSH (basic)
%    \end{itemize}
    
        \end{multicols}

%\hrule
%\vspace{.3 cm}
%\begin{itemize}[label=$\blacktriangleright$]
%\item Extensive knowledge of \emph{Python}  and \emph{Mathematica} 
%\item Complete familiarity with \LaTeX
%\item Familiarity with Matlab and Fortran %, and basic knowledge of R
%\end{itemize}

\subsection*{Idiomas}

%\hrule
%\vspace{.3 cm}

    \begin{multicols}{2}


\begin{itemize}[label=$\blacktriangleright$]
\item \emph{Espa\~{n}ol}: Lengua materna
\item \emph{Ingl\'es}: Nivel Avanzado
\item \emph{Franc\'es}: Nivel Intermedio (B1)
\end{itemize}

\end{multicols}


\section*{Desarrollo profesional}
\subsection*{Talleres}
\begin{itemize}[label=$\blacktriangleright$]

\item \textbf{Taller de la Alianza Nacional de Ciencia de Datos.}: An\'alisis de datos exploratorios (EDA) \& Visualizaci\'on de datos. marzo, 2024




\item \textbf{Taller de la Alianza Nacional de Ciencia de Datos}: Navegando por el panorama de la ciencia de datos: ¿qu\'e, por qu\'e y c\'omo? Enero, 2024


\item \textbf{Escuela de Verano EMIT en Astronom\'ia Multimensajero}: Conferencias y tutoriales en ondas gravitacionales, estad\'istica, astronom\'ia multimensajero y temas de diversidad e inclusi\'on. Verano, 2023

\item \textbf{Escuela de verano de Penn State en estadí\'istica para astr\'onomos}: Conferencias y tutoriales en estad\'istica, astronom\'ia e inform\'atica. Verano, 2021

\item \textbf{SciPy 2021:} Conferencia sobre Computaci\'on Cient\'ifica con Python. Verano, 2021
    \item  \textbf{Curso de Wikipedia cient\'ifico}: Elevando la visibilidad de los cient\'ificos de f\'isica cu\'antica en Wikipedia por parte de la Sociedad Estadounidense de F\'isica y Wiki Education. \href{https://wikiedu.org/blog/2021/05/24/improving-the-quality-of-spanish-language-articles-on-wikipedia/}{Publicaci\'on de blog: Mejorar la calidad de los art\'iculos en espa\~{n}ol en Wikipedia (Improving the quality of Spanish language articles on Wikipedia)}
    \item \textbf{ComSciCon:} El taller de comunicaci\'on de la ciencia para estudiantes de posgrado. Junio, 2020..
    %\item \textbf{SCOPE:} Science Communication Online Programme. Fall, 2020.
   % \item \textbf{Reclaiming STEM:} Diverse \& inclusive SciComm workshop. Fall, 2020.
    


            \item \textbf{Taller de preparaci\'on de propuestas del JWST:} Descripci\'on de instrumentos, herramientas de planificaci\'on y sesiones pr\'acticas. Primavera de 2020.

    
    
    
        \item \textbf{Taller del Observatorio Nacional de Radioastronom\'ia:}  Evento de dos d\'ias, que incluye sesiones pr\'acticas y preparaci\'on de observaci\'ones y reducci\'on de datos. Primavera de 2019.

\end{itemize}


\section*{Divulgaci\'on Cient\'ifica}
\hrule
\vspace{.3 cm}
\begin{itemize}[label=$\blacktriangleright$]

\item \emph{Author en \href{https://astrobitos.org/}{Astrobitos}}: Uno de los autores originales. \\ Astrobitos es la versi\'on en espa\~{n}ol de Astrobites.\\ URL: \url{https://astrobitos.org/author/manuelmarcano22/}


\subsection*{Charlas}


\item \textit{Binarias de Enanas Blancas.} \\
Charla. "Colegio Nuestra Se\~{n}ora De La Concepci\'on". \\
Colombia. March 2024.


\item \textit{Astronom\'ia Estelar} \\
Charla. "Mes de la Astronom\'ia en la Rep\'ublica Dominicana". \\Universidad Aut\'onoma de Santo Domingo. Verano 2023.


\item \textit{Binarias de Enanas Blancas: Sirenas de Ondas Gravitacionales y Laboratorios Estelares}. \\
Charla. Asociaci\'on Estadounidense de Observadores de Estrellas Variables. 2022.




\item \textit{Caminando en la Luna}. \\
Charla. Planetario de San Jos\'e Universidad de Costa Rica (UCR). 2021.


\end{itemize}



\subsection*{El Museo Americano de Historia Natural}

\begin{itemize}[label=$\blacktriangleright$]

	\item \textit{Parte del equipo traduciendo al espa\~{n}ol el curso "Aprendizaje Autom\'atico para F\'isica y Astronom\'ia"}. \\
\url{https://openlearning.flatironinstitute.org/courses/course-v1:cca+ML_01+A/about}

\end{itemize}





\subsection*{Universidad Tecnol\'ogica de Texas}

\begin{itemize}[label=$\blacktriangleright$]

	\item \textit{Una aguja en un pajar: encontrando binarias ex\'oticas en c\'umulos de estrellas}. \\
Astronight. YWCA Lubbock. 2022. \\
\url{https://slides.com/mmarcano22/ywca2022}


%\item Solar Observation at TexPREP-Lubbock (http://www.math.ttu.edu/texprep/) 

	\item \textit{Presentaci\'on del banner del 30.º aniversario del Hubble}. \\
Astronight. YWCA Lubbock. 2022. \\
\url{https://hubblesite.org/hubble-30th-anniversary/events}



	%\item \textit{"Hubble's 30th Anniversary/30 aniversario del telescopio Hubble"} \\ Won a competition to host one of the anniversary banners. \\ Bilingual talk (Spanish/English) for Astronight event at TTU, Lubbock, TX. 2020
	



%\item Solar Observation at TexPREP-Lubbock (http://www.math.ttu.edu/texprep/) 
	\item \textit{\href{https://slides.com/mmarcano22/a-tour-of-the-milky-way-2\#/}{"Un Tour por la V\'ia Lactea"}} \\ Charla para la noche de astronom\'ia. TTU, Lubbock, TX. 2019


	\item \textit{\href{http://manuelpm.me/PCASDSS/}{"Reducci\'on de dimensionalidad aplicada a grandes estudios espectrosc\'opicos"}} \\ Escuela Secundaria Emmy Noether. D\'ia de las Matem\'aticas, Lubbock, TX. 2018



%\item Science Made Simple at the TTU museum. 

%\item +10 Solar Observation and school visits to do science demos and judge science fair projects in Lubbock
\end{itemize}


\subsection*{Sociedad Astron\'omica Dominicana (Astrodom)}

\begin{itemize}[label=$\blacktriangleright$]


    \item \textit{\href{https://manuelpm.me/AstrodomJulio2020/}{"Una aguja en un pajar: Buscando remanentes estelares en c\'umulos globulares"}} \\ Santo Domingo, Rep. Dom. 2020


	\item \textit{\href{http://manuelpm.me/AstrodomGBStalk/}{"Galactic Bulge Survey: Fuentes de Rayos X del Bulbo Galactico"}} \\ Santo Domingo, Rep. Dom. 2018



	\item \textit{\href{http://manuelpm.me/Astrodomtalk/}{"Estudio de la poblaci\'on de objetos compactos en el c\'umulo globular NGC 6397 con la unidad de campo integral MUSE"}} \\Santo Domingo, Rep. Dom. 2016

\end{itemize}



%\section*{Awards \& Grants}
%\begin{itemize}
  %\item Rolf Winter Physics Teaching Assistant Award, 2007, 2008, 2009, 2010. 
  %\item VSGC Graduate Research Fellowship, 2009-2010.
  %\item Token travel grant from W\&M's grad dean. 
%\end{itemize}



%\section*{Professional Service}
%\begin{itemize}
% \item \textbf{Graduate Representative}, Physics Graduate Studies Committee, W\&M, 2009-2010.
% \item \textbf{President}, Physics Graduate Student Association, W\&M, 2002-2003.
%\end{itemize}
%
%\section*{Professional Development}
%\begin{itemize}
% \item \textbf{Generic Physics Summer School}, Fermilab, Summer 2002.
%\end{itemize}
%



\pagebreak


\section*{Referencias }  


   \begin{tabbing}
 \emph{Dr. Thomas Maccarone}\\
Associate Professor\\
Texas Tech University \\
Department of Physics and Astronomy \\
Lubbock TX 79409-1051 \\
PHONE:  1-806-742-3760 \,\,\,\,\,\,\,\,\, eMAIL: 
thomas.maccarone@ttu.edu
\end{tabbing}




   \begin{tabbing}
 \emph{Dr. Liliana Rivera Sandoval}\\
Assistant Professor \\
The University of Texas Rio Grande Valley \\
Department of Physics and Astronomy \\
BINAB 2.102 One West University Blvd. \\
Brownsville 78520 \\
PHONE:  +1 9568825131 \,\,\,\,\,\,\,\,\, eMAIL: 
liliana.riverasandoval@utrgv.edu
\end{tabbing}

   \begin{tabbing}
 \emph{Dr. Craig Heinke}\\
Professor \\
University of Alberta \\
Department of Physics \\
 CCIS 4-183 \\
 Edmonton, AB, T6G 2E1, Canada\\
PHONE:  (780) 222-4815\,\,\,\,\,\,\,\,\, eMAIL: 
 heinke@ualberta.ca
\end{tabbing}







\begin{comment}

   \begin{tabbing}
 \emph{Dr. Craig Heinke}\\
Professor \\
University of Alberta \\
Department of Physics \\
 CCIS 4-183 \\
 Edmonton, AB, T6G 2E1, Canada\\
PHONE:  (780) 222-4815\,\,\,\,\,\,\,\,\, eMAIL: 
 heinke@ualberta.ca
\end{tabbing}


   \begin{tabbing}
 \emph{Dr. David Zurek}\\
Data Collections Manager Physical Sciences, Astrophysics \\
American Museum of Natural History \\
New York, NY\\ 
PHONE:  212-313-7318  \,\,\,\,\,\,\,\,\, eMAIL: dzurek@amnh.org
\end{tabbing}




   \begin{tabbing}
 \emph{Dr. Beth Thacker}\\
Associate Professor \\
Texas Tech University \\
Department of Physics and Astronomy \\
Lubbock TX 79409-1051 \\
PHONE:  (806) 834-2996  \,\,\,\,\,\,\,\,\, eMAIL: 
Beth.Thacker@ttu.edu
\end{tabbing}


\end{comment}




\begin{comment}

\section*{References }  


   \begin{tabbing}
 \emph{Dr. Thomas Maccarone}\\
Associate Professor\\
Texas Tech University \\
Department of Physics and Astronomy \\
Lubbock TX 79409-1051 \\
PHONE:  1-806-742-3760 \,\,\,\,\,\,\,\,\, eMAIL: 
thomas.maccarone@ttu.edu
\end{tabbing}




   \begin{tabbing}
 \emph{Dr. Natalie Webb}\\
Researcher\\
Institut de Recherche en Astrophysique et Planétologie \\
9 Avenue de Colonel Roche \\
31028 Toulouse Cedex 4, France\\ 
PHONE:  (+33) 5 61 55 75 70  \,\,\,\,\,\,\,\,\, eMAIL: Natalie.Webb@irap.omp.eu
\end{tabbing}


\begin{tabbing}
 \emph{Dr. Martin Hendry}\\
Head of School\\
Professor of Gravitational Astrophysics and Cosmology \\
SUPA, School of Physics and Astronomy \\
University of Glasgow  \\
Glasgow, G12 9RW \\ 
PHONE:  (+44) 141 330 5685   \,\,\,\,\,\,\,\,\, eMAIL: Martin.Hendry@glasgow.ac.uk
\end{tabbing}


   \begin{tabbing}
 \emph{Dr. Shane Larson}\\
Research Associate Professor \\
Center for Interdisciplinary Exploration \\
and Research in Astrophysics \\
Northwestern University \\
Evanston, IL 60208  \\ 
PHONE:  (847) 467 4305   \,\,\,\,\,\,\,\,\, eMAIL: s.larson@northwestern.edu
\end{tabbing}







%\section{References }  
%   \begin{tabbing}
% \emph{Dr. Shane Larson}\\
%Assistant Professor of Physics \\
%Department of Physics \\
%4415 Old Main Hill \\
%Utah State University \\
%Logan, UT 84322-4415  \\ 
%PHONE:  435-797-8838  \,\,\,\,\,\,\,\,\, eMAIL: s.larson@usu.edu
%\end{tabbing}

  \begin{tabbing}
 \emph{Dr. Scott W. McIntosh}\\
National Center for Atmospheric Research\\
High Altitude Observatory \\
 3080 Center Green Drive - CG1	 \\
Boulder, CO 80301	  \\ 
PHONE:  (303) 497 1544  \,\,\,\,\,\,\,\,\, eMAIL: mscott@ucar.edu
\end{tabbing}



  \begin{tabbing}
 \emph{Dr. David Peak}\\
 Professor of Physics \\
Department of Physics \\
4415 Old Main Hill \\
Utah State University \\
Logan, UT 84322-4415  \\ 
PHONE:  (435) 797-2884  \,\,\,\,\,\,\,\,\, eMAIL: david.peak@usu.edu.
\end{tabbing}






%\bigskip
% Footer
%\begin{center}
  %\begin{footnotesize}
   % Last updated: \today \\
%    \href{\footerlink}{\texttt{\footerlink}}
%  \end{footnotesize}
%\end{center}
\end{comment}

\end{document}





\end{document}

